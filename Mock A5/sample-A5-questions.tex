\documentclass{article}
\usepackage[utf8]{inputenc}
\usepackage{amsmath}
\usepackage{amsfonts}
\usepackage{amssymb}
\usepackage{geometry}
\usepackage{enumitem}

\geometry{margin=1in}
\setlength{\parindent}{0pt}

\title{CS 338 Fun Assignment \#5}
\author{}
\date{}

\begin{document}

\maketitle

\section*{Question 1.}
Based on the given database scenario and concurrent transactions, 
determine the appropriate isolation levels.

Consider an online banking system with the following database schema:
\begin{itemize}
    \item Account(account\_id, customer\_id, balance, account\_type)
    \item Transaction\_Log(log\_id, account\_id, transaction\_type, amount, timestamp)
\end{itemize}


For each of the following transaction scenarios, determine the 
\textbf{lowest appropriate isolation level} and explain your reasoning:\\

\textbf{(a)} Transaction T1 performs a single account balance update:
\begin{verbatim}
-- T1:
UPDATE Account 
SET balance = balance + 500 
WHERE account_id = 12345;
COMMIT;
\end{verbatim}

\textbf{(b)} Transaction T2 generates a monthly account statement:
\begin{verbatim}
-- T2:
SELECT customer_id, balance, account_type 
FROM Account 
WHERE customer_id = 67890;

SELECT log_id, transaction_type, amount, timestamp 
FROM Transaction_Log 
WHERE account_id IN (
    SELECT account_id FROM Account WHERE customer_id = 67890
);
COMMIT;
\end{verbatim}


\textbf{(c)} Transaction T3 calculates and updates interest for all savings accounts:
\begin{verbatim}
-- T3:
SELECT account_id, balance 
FROM Account 
WHERE account_type = 'SAVINGS';

-- For each account found, calculate 2% interest
UPDATE Account 
SET balance = balance * 1.02 
WHERE account_type = 'SAVINGS';

-- Log the interest transactions
INSERT INTO Transaction_Log 
SELECT NEXTVAL('log_seq'), account_id, 'INTEREST', balance * 0.02, NOW()
FROM Account 
WHERE account_type = 'SAVINGS';
COMMIT;
\end{verbatim}

\newpage
\section*{Question 2.}

Conceptual questions on concurrency anomalies and storage search method.\\

\textbf{(a)} Explain the difference between an \textbf{unrepeatable read} 
and a \textbf{phantom read}. \\

\textbf{(b)} Suggest 1 \textbf{benefits} and 1 \textbf{potential problems} of maintaining 
secondary indexes in a high‐update 
OnLine Transaction Processing workload (systems that handle a large number of short, 
fast transactions in real time, onine banking for example).\\

\newpage
\section*{Question 3.}
Design and analyze a database system that handles both transaction isolation and 
indexing requirements.

You are designing a library management system with the following tables:
\begin{itemize}
    \item Book(isbn, title, author, category, copies\_available)
    \item Member(member\_id, name, email, membership\_type)
    \item Checkout(checkout\_id, member\_id, isbn, checkout\_date, due\_date, 
    returned\_date)
\end{itemize}

\textbf{(a)} The system needs to handle the following concurrent transaction. 
Determine the minimum isolation level required.\\

A member checks out a book
\begin{verbatim}
-- T_CHECKOUT:
SELECT copies_available FROM Book WHERE isbn = '978-0123456789';
-- If copies_available > 0:
UPDATE Book SET copies_available = copies_available - 1 
WHERE isbn = '978-0123456789';
INSERT INTO Checkout 
VALUES (NEXTVAL('checkout_seq'), 12345, '978-0123456789', 
        CURRENT_DATE, CURRENT_DATE + 14, NULL);
COMMIT;
\end{verbatim}

Consider that multiple members might try to check out the last copy of the same book 
simultaneously.\\


\textbf{(b)} Consider the following two queries that are frequently executed on the 
library management system:\\

T1:
\begin{verbatim}
SELECT * FROM Book WHERE isbn = '978-0123456789';
\end{verbatim}

T2:
\begin{verbatim}
SELECT b.title, b.author, c.checkout_date, c.due_date
FROM Book b, Checkout c
WHERE b.isbn = c.isbn 
  AND c.returned_date IS NULL
  AND c.member_id = 12345;
\end{verbatim}

For each query, design an optimal indexing strategy.

\end{document}